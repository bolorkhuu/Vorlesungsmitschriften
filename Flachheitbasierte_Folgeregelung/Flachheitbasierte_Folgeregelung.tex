\documentclass[ngerman]{tudscrreprt}
\usepackage{selinput}
\SelectInputMappings{adieresis={ä},germandbls={ß}}
\usepackage[T1]{fontenc}
\usepackage{babel} \usepackage{isodate}
\usepackage{amssymb}
\usepackage{amsmath}
\usepackage{float} % lädt das Paket zur Verwendung von zusätzlichen Positionsbefehlen
\usepackage{wrapfig}  
\usepackage{picinpar}   
\usepackage{hyperref}

\begin{document}
\faculty{Fakultät Elektrotechnik und Informationstechnik} \department{} \institute{INSTITUT FÜR REGELUNGS- UND STEUERUNGSTHEORIE} \chair{} \title{Flachheitsbasierte Folgeregelung\\Dr.-Ing. F. Woittennek
}
% \thesis{diss}
% \degree[Dr.-Ing.]{Doktor-Ingenieur}
\author{Mitschrift von Bolor Khuu}
% \dateofbirth{2.1.1990}
% \placeofbirth{Dresden}
\date{\today}
% \defensedate{20.10.2014}
% \referee{Dagobert Duck \and Mac Moneysac}
\maketitle
\tableofcontents
\newpage
\chapter{Einführung}
\subsection*{Kinematisches Modell eines zweiachsichen Fahrzeuges}
Beschreibung der Lage
\begin{itemize}
\item Position Hinterachsenmittelpunkt H:
\begin{equation*}
\underline{y} = y_1 \underline e_1 + y_2 \underline e_2
\end{equation*}
\item Position des Vorderachsenmittelpunkts:
\begin{equation*}
\underline{y}_v = y_{1,v} \underline e_1 + y_{2,v} \underline e_2
\end{equation*}
\item Ausrichtung des Fahrzeuges:
\begin{equation*}
\underline{\tau} = \underline e_1 \cos{\theta} + \underline e_2 \sin{\theta}
\end{equation*}
\item zugerhörige Normalenvektor: 
\begin{equation*}
\underline{\nu} = -\underline e_1\sin{\theta} + \underline e_2 \cos{\theta} = \frac{d\underline{\tau}}{d\theta}
\end{equation*}
\item Ausrichtung der Vorderräder:
\begin{equation*}
\underline{\tau}_\nu = \cos{\theta + \varphi}\underline e_1 + \sin{\theta + \varphi} \underline e_2 = \cos{\varphi} \underline{\tau} + \sin{\varphi}\underline{\nu}
\tag{1}
\end{equation*}
\item zugerhöriger Normalenvektor:
\begin{equation*}
\underline{\nu}_v = -\sin{\varphi}\underline{\tau} + \cos{\varphi} \underline \nu
\label{Eq:norma2}
\tag{2}
\end{equation*}
\end{itemize}
Ausrichtung der Bindungen
\begin{itemize}
\item Vorderachsenmittelpunkt ist starr mit Hinterachse verbunden (Abstand $l$)
\begin{equation*}
\underline{y}_v = \underline y + l \underline \tau
\label{Eq:vord}
\tag{3}
\end{equation*}
\item Hinterräder rollen, ohne zu gleiten:
\begin{equation*}
\dot{\underline{y}} = v \underline \tau 
\tag{4}
\end{equation*}
\item Differenzieren von \eqref{Eq:vord}:
\begin{equation*}
\dot{\underline{y}}_v = \dot{\underline y} + l \underline \tau = v \underline \tau + l \underline \nu \dot{\theta}
\tag{5}
\end{equation*}
\item Vorderräder rollen, ohne zu gleiten: 
\begin{equation*}
\langle \underline{\nu}_v, \dot{\underline{y}}_v \rangle = 0
\end{equation*}
\item aus \eqref{Eq:norma2} folgt unter Beachtung von 
\begin{equation*}
\langle \underline{\nu},\underline{\tau}\rangle = \langle \underline \tau , \underline \nu \rangle = 0
\end{equation*}
\begin{equation*}
\langle \underline{\tau},\underline{\tau}\rangle = \langle \underline \nu , \underline \nu \rangle = 1
\end{equation*}
der Zusammenhang: 
\begin{equation*}
\begin{matrix}
\langle \nu_v , \dot{y}_v \rangle &= &\langle -\sin{\varphi}\underline {\tau} + \cos{\varphi}\underline \nu , v\underline \tau + l \underline \nu \dot{\theta}\rangle\\ 
&=& \langle -\sin{\varphi \underline \tau , v \underline \tau \rangle + \langle -\sin{\varphi}}\underline{\tau}, l \underline \nu \dot{\theta} \rangle + \langle \cos{\varphi}\underline \nu , v \underline \tau \rangle + \langle \cos{\varphi}\underline \nu , l \underline \nu \dot{\theta} \rangle \\ 
&=& -v \sin{\varphi} + \cos{\varphi} l \dot{\theta}
\end{matrix}
\tag{6}
\end{equation*}
\item Zusammenfassung Modell: 
\begin{equation*}
\begin{matrix}
\dot y_1 &=& v\cos{\theta}& \Leftrightarrow \dot{\underline y} = v\underline \tau\\ 
\dot y_2 &=& v\sin{\theta}&
\end{matrix}
\tag{7a}
\end{equation*}
\begin{equation*}
\tan{\varphi} = \frac{l\dot{\theta}}{v}
\tag{7b}
\end{equation*}
\end{itemize}
\section*{Flachheit des Fahrzeugmodells}
Ziel: Berechnung alles Systemgrößen $(y_1, y_2, \varphi, \theta, v)$ aus Bahn des Hinterachsenmittelpunktes 
\begin{itemize}
\item Geschwindigkeit aus (7a): \\ 
\begin{equation*}
|v| = |v| |\underline \tau| = |\dot{\underline y}| = \sqrt{{\dot{\underline y}_1}^2 + {\dot{\underline y}}_2^2} \Rightarrow v = \pm |\dot{\underline y}|
\tag{8}
\end{equation*}
\item Orientierung: 
\begin{equation*}
\underline \tau = \frac{\dot{\underline y}}{v} = \frac{\dot{\underline y}}{\pm |\dot{\underline y}|}
\tag{9}
\end{equation*}
\item zugerhöriger Winkel:
\begin{equation*}
\tan{\theta} = \frac{\dot y_2}{\dot y_1}
\tag{10}
\end{equation*}

\item Differenzieren (7a) liefert: 
\begin{equation*}
v \dot{\theta}\underline \nu + \dot {v} \underline \tau = \ddot{\underline y}
\tag{11}
\end{equation*}
\item skalare Multiplikation mit $v, \underline \nu = - \dot{y}_2 \underline e_1 + \dot y_1 \underline e_2$ 
\begin{equation*}
v^2 \dot \theta = \langle \ddot{\underline y}, -\dot y_2 \underline e_1 + \underline{\dot{y}}_1 e_2 \rangle = -\ddot{y}_1 \dot{y}_2 + \ddot y_2 \dot y_1 \Rightarrow \dot \theta = \frac{\ddot y_2 \dot y_1 - \ddot y_1 \dot y_2}{{\dot y_1}^2 + {\dot y_2}^2}
\tag{12}
\end{equation*}
\item Lenkwinkel: 
\begin{equation*}
\tan{\varphi} = l \frac{\ddot y_2 \dot y_1 - \ddot y_1 \dot y_2}{\pm{({\dot y_1}^2 + {\dot y_2}^2)}^{3/2}}
\tag{13}
\end{equation*}
\end{itemize}
\section*{Flachheitsbasierter Steuerungsentwurf für das Fahrzeug} 
Vorüberlegungen:
\begin{itemize}
\item Wahl einer Übergangszeit $t_f$
\item Wahl von Anfangs- und Endbedingungen 
\begin{equation*}
(y_1(0), y_2(0), \theta(0), \varphi(0), v(0), \dot{v}(0)), (y_1(t_f), y_2(t_f), \theta(t_f),\varphi(t_f), \dot v (t_f))
\end{equation*}
\item Berechnung der zugerhörigen Bindungen für flachen Ausgang aus (7a) und (11)
\begin{equation*}
\begin{matrix}
\dot{\underline y} &=& v \underline \tau &=& v(\underline e_1 \cos{\theta} + \underline e_2 \sin{\theta})&\\ 
\ddot{\underline y}&=& \dot v \tau + v \dot \theta \nu &=& \dot v (\underline e_1 \cos{\theta} + e_2 \sin{\theta} ) + \frac{v^2 \tan{\varphi}}{l}& \qquad (-\sin{\theta} \underline e_1 + \cos{\theta} \underline e_2)
\end{matrix}
\end{equation*}
\item Vorgabe der Solltrajektorie für den flachen Ausgang als zweimal differenzierbare Zeitfunktion $t \to \underline y_d(t)$ unter Berücksichtigung der Anfangs- und Endbedingungen
\item Steuerung: Berechnung der Zeitverläufe aller Systemgrößen (u.a. Stellgrößen aus $t \to y_d (t)$)
\end{itemize}
\section*{Flachheitsbasierter Folgeregler}
\subsection*{Problem} Anfangsfehler , Störung, ungenau Modellparameter, nicht modellierte Modelleigenschaften (z.B. Schlupf) verursachen Abweichungen von geplanten Bahn 
\subsection*{Lösung} Regler zur Zurückführung des Fahrzeuges auf geplanten Bahn (Stabilisierung der Folgebewegung)
\subsection*{Annahme} Position und Orientierung des Fahrzeuges sind bekannt
\subsection*{Folgefehler} 
% \begin{equation*} 
% \underline{e} = 
% \begin{pmatrix} e_1(t)\\ 
% e_2(t) 
% \end{pmatrix} = 
% \begin{pmatrix} 
% y_1(t)- y_{1,1(t)\\ 
% y_2(t)- y_{2,d}(6) 
% \end{pmatrix}
% \end{equation*}
\subsection*{Ziel} Folgefehler soll abklingen: $\lim_{t\to \infty} e_i(t)= 0, \quad i =1,2$
Ziel erreicht, wenn Folgefehler sich wie stabile Oszillatoren verhalten. 
\begin{equation*}
\ddot{e}_i(t) + 2\delta_i \omega_i \dot{e}_i(t) + \omega_i^2 e_i(t) = 0 \quad i =1,2,\quad \delta_i \omega_i > 0 
\tag{15}
\end{equation*}
Gleichung (15) kompakt: 
\begin{equation*}
\underline{\ddot{e}}(t) + K_1 \underline{\dot{e}}(t) + K_0 \underline{e}(t) = 0, \qquad K_1 = \text{diag}(2\delta_1\omega_1, 2\delta_2\omega_2), K_0 = \text{diag}(\omega_1^2, \omega_2^2)
\end{equation*}
\subsection*{Frage}: Wie müssen Stellgrößen gewählt werden, von um (15) zu erreichen? 
\begin{itemize}
\item aus (14) und (15) folgt: 
\begin{equation*}
\ddot{y}_(t)= \underline{\ddot{y}}_d(t) - K_1\underline{\dot{e}}(t) - K_0 \underline{e}(t) = \underline{\ddot{y}}_d - K_1(v(t)\underline{\tau}(t) - \underline{\dot{y}}_d(t)) 
\end{equation*}
fehlt etwas
\item 
Auflösen nach $\dot v$ bzw. $v \dot{\theta}$ durch skalare Multiplikation mit $\underline{v}$bzw.  $\underline{\tau}$ 
\begin{equation*}
\dot v = \langle \underline{\ddot y}_d - K_1 (v \underline \tau - \underline{\dot y}_d ) - K_0 (\underline y - \underline y_d), \underline{\tau} \rangle \tag{16a}
\end{equation*}
\begin{equation*}
v \dot{\theta} = \rangle \underline{\dot{y}}_d - K_1(v\underline{\tau}- \underline{\dot{y}}_d) -K_0(\underline y - \underline y_d), \underline \nu \rangle \tag{16b}
\end{equation*}
\item (16a) ist DGL für Stellgröße $v$ (Anfangsbedingungen für $v$ aus Solltrajektorie)$\Rightarrow$ dynamischer Regler
\item (16b) liefert mit (13) den Lenkwinkel. 
\begin{equation*}
\varphi = \text{arctan}(  l\frac{\langle \underline{\ddot{y}}_d - K_1(v\underline{\tau} - \underline{\dot y}_d - K_0 (\underline y - \underline{y}_d)), \underline{\nu}  \rangle}{v^2})
\end{equation*}
\item Regler aus (16) (17a) mit Anfangsbedingung $v_0 = v_d$
\end{itemize}
\subsection*{Problem} Singularität bei $v=0$ vorhanden Steuerung /Regelung des Fahrzeuges in/aus Ruhelagen (Abschnitt 3)
\section*{Zusammenfassung:}\begin{itemize}
\item Entwurf eines Folgereglers in zwei Schritten: Plannung einer Solltrajektorie und Stabilisierung entlang der Solltrajektorie
\item Verwendung eines flachen Ausgangs erleichtert beide Schritte beschriebenes Vorgehen ist für flache Systeme generell möglich 
\end{itemize}
\chapter{Flache Systeme}
Ausgangspunkt: System von $q$ gewöhnlichen , nichtlinearen DGLen in den Systemgrößen $z = (z_1,\dots, z_s)$ 
\begin{equation*}
S_i (z_i,\dots, z^{\sigma_i}) = 0, \quad i=1,\dots,q \tag{1}
\end{equation*}
Spezialfall nichtlineares Zustandsgleichungenssystem: 
\begin{equation*}
\underline{\dot{x}} = \underline{f}(\underline{x},\underline{u}),\quad \underline{x}\in \mathrm{R}^u, u\in \mathrm{R}^m
\end{equation*}
in der Form (1) $z=(x_1, \dots, x_n, u_1,\dots, u_m)$
\begin{equation*}
S_i(z,\dot z) = \dot{x}_i - f_i(\underline{x},\underline{u}),\qquad i=1,\dots, n
\end{equation*}
\section{Definition der flachen Systeme}
Interpretation:
\begin{itemize}
\item{(I)}. Trajektorien für $y$ können frei gewählt werden
\item{(II)}. Berechnung der Zeitverläufe der Systemgrößen aus jenen von $y$ (ohne Integration) möglich 
\end{itemize}
\subsection{Unendlich viele Ausgänge}
\begin{itemize}
\item flacher Ausgang nicht eindeutig 
\item Umrechnung zweier flacher Ausgänge $\tilde y$ und $y$ mittel (II)
\begin{equation*}
\begin{matrix}
\tilde y_i &=& \psi_i(y, \dot{y}),&\quad& i = 1,\dots, \tilde m\\ 
y_i &=& \tilde{\psi}_i(\tilde y, \tilde{y}, \dots)&\quad& i =1,\dots, m
\end{matrix}
\end{equation*}
\item man kann zeigen, dass $\tilde m = m$ (jeder flache Ausgang hat gleiche Anzahl an Komponenten)
\end{itemize}
\subsection{Zur differentiellen Unabhängigkeit der flachen Ausgangs}
\begin{itemize}
\item Überprüfung der differentiellen Unabhängigkeit im allgemein schwierig
\item sei $y=(y_1, \dots, y_m)$ mit $m = s-q$ ein $m$-Tupel, das $(II)$ genügt
\item Transformation $\Phi$: $(z_1, \dots, z_s) \to (\bar z_1,\dots, \bar z_s)$
\begin{equation*}
\begin{matrix}
\bar z_1 &=& y_1 &=& \phi_1(z,\dots, z^{\beta_1})\\ 
\vdots &&&&\\
\bar z_m &=& y_m &=&\phi_m(z, \dot z, \dots,z^{\beta_m})\\
\bar z_{m+1}&=& &&S_1(z, \dot z, \dots) = 0\\ 
\vdots &&&&\\
\bar z_{s} &=& && S_q(z, \dot z, \dots) = 0  
\end{matrix}
\end{equation*}
mit Rücktransformation: 
\begin{equation*}
\begin{matrix}
z_1 &=& \psi_n(y,\bar y,\dots) \\ 
\vdots &&\\ 
z_s &=& \psi_s(y, \dot y,\dots)
\end{matrix}
\end{equation*}
\item in neuen Koordinaten $\bar z$ gilt: \begin{equation*}
\bar z_{m+1} = \dots = \bar z_s = 0 \tag{2} 
\end{equation*}
$(y_1,\dots, y_m) = (\bar z_1, \dots, \bar z_m)$ kommt in (2) nicht vor $\Rightarrow$ differentielle Unabhängigkeit offensichtlich 
\end{itemize}
\subsection*{Folgerung}
\begin{itemize}
\item{1.} Ein flacher Ausgang eines durch $q$ (unabhängige Gleichung gegebenen Systems für $s$ Größen hat stets $m=s-q$ Komponenten.)
\item{2.} Genügt $y= (y,\dots,y_m)$ der Bedingung $(II)$, so ist Bedingung $(I)$ automatisch erfüllt.
\end{itemize}
\subsection{Ebenes Modell eines Brückenkrans}
\begin{itemize}
\item Modell mit Systemgrößen: $y_1,y_2, D_2, R, \theta, T, \omega, C,F$ 
\begin{itemize}
\item Impulserhaltung für Last. 
\begin{equation*}
m(\underline{\ddot{y}} - \underline{g}) = -T\underline{\tau} \tag{3a}
\end{equation*}
\item geometrischer Zussamenhang zwischen Wagen und Lastposition 
\begin{equation*}
\underline y = D_2 \underline{e}_2 + R\underline{\tau} \tag{3b}
\end{equation*}
\item Impulserhalung für Wagen
\begin{equation*}
M\ddot{D}_2 = F + T\sin{\theta} - c_d \dot{D}_2, \qquad c_d \dots\text{Reibkoeffizient} \tag{3c}
\end{equation*}
\item Drehimpulsbilanz Rolle
\begin{equation*}
J\dot{\omega} = C - \rho T - c_r \omega,\qquad c_r\dots\text{Reibkoeffizient}\tag{3d}
\end{equation*}
\item Zusammenhang Seillänge Winkelgeschwindigkeit 
\begin{equation*}
\dot R = -\rho \omega \tag{3e}
\end{equation*}
mit 
\begin{equation*}
\underline{\tau} = \cos{\theta}\underline{e}_1 + \sin{\theta}\underline{e}_2,\qquad \underline{\nu} = -\sin{\theta}\underline{e}_1 + \cos{\theta}\underline{e}_2 \tag{4}
\end{equation*}
\end{itemize}
\item Pendelteilsystem besteht aus (3a) und (3b) mit Größen $R,\theta, y_1,y_2,T,D_2$ besitzt gleichen flachen Ausgang wie Gesamtsystem
\end{itemize}
\end{document}